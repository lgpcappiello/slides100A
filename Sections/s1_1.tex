\begin{frame}{Case Study: Using Stents to Prevent Strokes}
    A classic challenge in statistics is evaluating the efficacy of medical treatments. 
    \begin{itemize}
        \item Stents are medical devices used to assist patients after cardiac events like strokes.
        \item Suppose we want to know if stents are also beneficial in helping to \textit{prevent} strokes.
        \item We start by writing our principal question: \\ \begin{center}
            Does the use of stents reduce the risk of stroke?
        \end{center}
        \item Now we can gather data to answer this question. 
    \end{itemize}
\end{frame}

\begin{frame}{Case Study}
    Some researchers conducted a study with 451 at-risk patients. Each patient was randomly assigned to either treatment (preventative stent) or control (no stent).
\end{frame}

\begin{frame}{Case Study}
    I do research in this area! 
    \begin{itemize}
        \item Usually when we test medical treatments, we do a randomized control trial. Basically, we get a sample of the population and randomly assign them to treatment or placebo. Then, we examine the difference in outcomes.
        \item RCTs have a lot of constraints that exclude certain individuals or occasionally make this kind of experiment unethical. 
        \item Because RCTs are sometimes not an options or certain groups of people are excluded, we might not be able to generalize our results.
        \item My research focuses on examining ways to use patient characteristics to estimate how these results might generalize.
    \end{itemize}
\end{frame}
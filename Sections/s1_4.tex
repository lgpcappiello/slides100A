\begin{frame}{Experiments}
    Studies where researchers assign treatments to cases are \textbf{experiments}. 
    
    \vspace{12pt}
    Whenever an experiment utilizes randomly assigned treatments, we say that it is a \textbf{randomized experiment}. 
    
    \begin{itemize}
        \item Note: "treatment" refers to whatever explanatory variable we are most interested in. 
    \end{itemize}
\end{frame}

\begin{frame}{Principles of Experimental Design}
    Four key principles:
    \begin{enumerate}
        \item Controlling
        \item Randomization
        \item Replication
        \item Blocking
    \end{enumerate}
\end{frame}

\begin{frame}{Principles of Experimental Design: Controlling}
    When treatments are assigned to cases, researchers do their best to \textbf{control} any other differences in the treatment groups. 
    \begin{itemize}
        \item For example, if both groups are given a pill to take with water, we might instruct everyone to drink the full 8 oz of water in order to control for any impact of water consumption.
    \end{itemize}
\end{frame}

\begin{frame}{Principles of Experimental Design: Randomization}
    Researchers \textbf{randomize} cases into treatment groups.
    \begin{itemize}
        \item This helps account for any unmeasured variables.
        \item For example, if we're studying a new cancer therapy and dog ownership has a positive impact on cancer outlook, randomization helps ensure that we have similar numbers of dog owners in each treatment group. 
        \item This helps minimize bias in our data. 
    \end{itemize}
\end{frame}

\begin{frame}{Principles of Experimental Design: Replication}
    The more information we have, the more confident we can be in our results! We gather more information through \textbf{replication}. 
    \begin{itemize}
        \item Suppose we have 3 treatment groups. Replication is just testing each treatment multiple times (multiple cases are assigned to each treatment group). 
    \end{itemize}
\end{frame}

\begin{frame}{Principles of Experimental Design: Blocking}
    If we suspect (or know) that other variables are important in influencing a response, we can group cases into \textbf{blocks}.
    \begin{itemize}
        \item Cases within each block are then randomly assigned to each treatment. 
        \item For example, if we are looking at a new asthma medication, we might block individuals by high, medium, and low severity of asthma. Then half of the individuals in each block would be assigned to the new medication. 
        \item This helps ensure that each treatment group has similar numbers of patients from each severity level. 
    \end{itemize}
\end{frame}

\begin{frame}{Principles of Experimental Design}
    \begin{itemize}
        \item All experiments will use some form of controlling, randomization, and replication. 
        \item Blocking is a slightly more advanced technique (in that it requires slightly more advanced methods to analyze). 
        \item You will learn more about blocking if you take STAT 100B. 
    \end{itemize}
\end{frame}

\begin{frame}{Bias in Human Experiments}
    Randomized experiments are the gold standard, but even they have their limitations!
    
    \vspace{12pt}
    Experiments involving people are especially prone to bias.
\end{frame}

\begin{frame}{Example: Heart Attack Drugs}
    Suppose we are interested in whether a new drug helps to prevent repeated cardiac events in patients who have already had at least one heart attack. 
    
    \begin{itemize}
        \item We get a random sample of 100 people who have had a heart attack in the past. 
        \item 50 of them are randomly assigned to the treatment (our new drug). This is our \textbf{treatment group}.
        \item The other 50 do not receive the drug. This is our \textbf{control group}. 
    \end{itemize}
    
    Can you think of anything that could bias our results?
\end{frame}

\begin{frame}{Sources of Bias}
    \begin{itemize}
        \item People who get the new drug expect it to work.
        \item E.g., people who did not get the drug may wonder if their study participation was worth the risk.
        \item Doctors may inadvertently affect the results through their optimism (or lack thereof) when administering the drug. 
    \end{itemize}
\end{frame}

\begin{frame}{Reducing Bias in Human Experiments}
    We can reduce bias by 
    \begin{itemize}
        \item Keeping patients uninformed about their treatment group.
        \begin{itemize}
            \item We call these studies \textbf{blind}.
            \item One way to keep studies blind is to give the control group a \textbf{placebo}.
        \end{itemize}
        \item Keeping doctors uninformed about which treatment groups their patients are in.
        \begin{itemize}
            \item We call these studies, where neither patient nor medical provider know the treatment group, \textbf{double-blind}. 
        \end{itemize}
    \end{itemize}
    
    \vspace{12pt}
    Can you think of some ethical issues that may arise in randomized, double-blind, placebo-controlled studies?
\end{frame}
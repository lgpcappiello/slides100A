\begin{frame}{Example: Insurance Deducibles}
    \begin{itemize}
        \item Suppose a health insurance company found that 70\% of the people they insure stay below their deductible in any given year.
        \item Each of these people can be thought of as a single trial in a study.
        \item We label a person a "success" if their healthcare costs do not exceed the deductible.
        \begin{itemize}
            \item $P(\texttt{success})=p=0.7$
            \item $P(\texttt{failure})=1-p=0.3$
        \end{itemize}
    \end{itemize}
\end{frame}

\begin{frame}{The Bernoulli Distribution}
    \begin{itemize}
        \item When an individual trial only has two possible outcomes it is called a Bernoulli random variable.
        \begin{itemize}
            \item These outcomes are often labeled as success or failure.
        \end{itemize}
        \item \textit{These labels can be completely arbitrary!}
        \begin{itemize}
            \item We called "not hitting the deductible" a "success", but we could just as well have labeled that the "failure".
            \item The framework we use to talk about the Bernoulli distribution does not depend on the label we use.
        \end{itemize}
    \end{itemize}
\end{frame}

\begin{frame}{The Bernoulli Distribution}
    Bernoulli random variables are often denoted as \texttt{1} for a success and \texttt{0} for a failure. 
    
    \vspace{12pt}This makes data entry easy and is mathematically convenient. 
    
    \vspace{12pt}Suppose we observe ten trials:
    \[
        1, \quad 1, \quad 1, \quad 0, \quad 1, \quad 0, \quad 0, \quad 1, \quad 1, \quad 0
    \]
\end{frame}

\begin{frame}{The Sample Proportion}
    The \textbf{sample proportion}, $\hat{p}$, will be the sample mean for these observations:
    \begin{align*}
        \hat{p} &= \frac{\text{\# of successes}}{\text{\# of trials}} \\
        \\
        &= \frac{1 + 1 + 1 + 0 + 1 + 0 + 0 + 1 + 1 + 0 }{10} \\
        \\
        &= 0.6
    \end{align*}
\end{frame}

\begin{frame}{The Bernoulli Random Variable}
    \begin{itemize}
        \item It is useful to think about a Bernoulli random variable as a random process with only two outcomes: a success or failure (or yes/no). 
        \item Then we code a success as 1 and a failure as 0.
        \item These are just numbers, so we can define the mean and variance.
    \end{itemize}
\end{frame}

\begin{frame}{The Bernoulli Random Variable}
    If $X$ is a random variable that takes the value 1 with probability of success $p$ and 0 with probability $1 − p$, then $X$ is a \textbf{Bernoulli random variable} with mean 
    \[
        \mu = p
    \]
    and variance
    \[
        \sigma^2 = p(1-p).
    \]
\end{frame}

\begin{frame}{The Bernoulli Distribution}
    \begin{itemize}
        \item Remember that we can estimate $p$ using $\hat{p}=\bar{x}$.
        \item We can use this to estimate the mean the variance.
        \item For our insurance deductible example, we found $\hat{p}=0.6$
        \item So we can estimate 
            
    \end{itemize}
    \[
        \hat{\mu} = \hat{p} = 0.6
    \]
    \hspace{1cm} and
    \[
        \hat{\sigma^2} = \hat{p}(1-\hat{p}) = 0.6*0.4 = 0.24
    \]
\end{frame}

\begin{frame}{Example}
    Derive the mean and variance of a Bernoulli random variable.
\end{frame}

\begin{frame}{Example}
    Because there are only 2 possible outcomes, the Bernoulli distribution describes a \textit{discrete random variable}. 
    
    \vspace{12pt}Therefore, We can start with its probability distribution table:
    \begin{center}
        \begin{tabular}{| l | c c |}
            \hline
            $x$ & $0$ & $1$ \\ \hline
            $P(x)$ & $p$ & $(1-p)$ \\ \hline
        \end{tabular}
    \end{center}
\end{frame}

\begin{frame}{Example}
    Then for the expected value,
    \begin{center}
        \begin{tabular}{| l | c c | c |}
            \hline
            $x$ & $1$ & $0$ & Total\\ \hline 
            $P(x)$ & $p$ & $(1-p)$ &  \\ \hline
            $xP(x)$ & $p$ & $0$ & $p$ \\ \hline
        \end{tabular}
    \end{center}
    
    \vspace{12pt} So the expected value is (as expected) $p$!
\end{frame}

\begin{frame}{Example}
    And for the variance,
    \begin{center}
        \begin{tabular}{| c | c c | c |}
            \hline
            $x$ & $1$ & $0$ & Total\\ \hline 
            $P(x)$ & $p$ & $(1-p)$ &  \\ \hline
            $xP(x)$ & $p$ & $0$ & $p$ \\ \hline
            $x-E(x)$ & $1-p$ & $-p$ &  \\ \hline
            $[x-E(x)]^2$ & $(1-p)^2$ & $p^2$ & \\ \hline 
            $P(x)[x-E(x)]^2$ & $p(1-p)^2$ & $(1-p)p^2$ & $p(1-p)^2 + (1-p)p^2$ \\
            \hline
        \end{tabular}
    \end{center}
\end{frame}

\begin{frame}{Example}
    Then
    \begin{align*}
        Var(X) &= p(1-p)^2 + (1-p)p^2 \\
        &= p-2p^2+p^3 + p^2-p^3 \\
        &= p-2p^2+p^2 \\
        &= p-p^2 \\
        &= p(1-p)
    \end{align*}
    \vspace{12pt}Which is the Var(X) we wanted!
\end{frame}